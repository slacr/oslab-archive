\documentclass{article}

\begin{document}

\title{Plato Grant Proposal}
\author{the ACRL : Our names here?}
\maketitle

\tableofcontents
\newpage 
\section{Goals and Needs}
\textit{What do you want to do?  What will you need?}
\linebreak 
\linebreak 
In the past, computer science research at TESC has been constrained by limited faculty
resources, demanding faculty schedules, and a lack of a consistent pool of students capable
of and willing to work on advanced topics. Coincidentally, this year, during the program
Student Originated Software, a number of advanced students have formed a nucleus of
student researchers that have the following attributes:
\linebreak 
they are advanced students and so are capable of working independently on
advanced topics
\linebreak 
collectively, they will be at TESC for another year beyond this year (giving us a two
year window in which to establish a stable base of software, hardware, and
expertise)
\linebreak 
their interests are in deploying practical systems based on the theoretical and
academic work they have been involved in during Computability and Student
Originated Software: they are interested in finding clients at Evergreen with whom
they can work to create computational solutions to domain dependent problems.
\linebreak 
\linebreak 

This represents a unique opportunity for us. With a stable group of students over 2 years,
we have the possibility of growing a significant expertise that can be passed on to less
experienced students as they enter the advanced programs and prepare to do student
research themselves. With such a group we will not experience the loss of knowledge and
capability typically found when we get the next generation of advanced students and the
previous generation graduates. They are committed to taking the theoretical topics we've
been studying and applying them to new domains, providing a basis for interdisciplinary
work at Evergreen that includes Computer Science in a fundamental way, rather than
merely as a tool or an application.
\linebreak
\linebreak


We have also been the recipient of a number of computers donated through St Peter's
hospital: one of our students works at St. Peter's and he arranged for the donation of a
number of computers that were being retired. Academic Computing has made a small lab
available for our dedicated use near the existing general purpose labs (the ACC and the
GCC).
\linebreak 
\linebreak 
This represents a unique opportunity for us. With a stable group of students over 2 years,
we have the possibility of growing a significant expertise that can be passed on to less
experienced students as they enter the advanced programs and prepare to do student
research themselves. With such a group we will not experience the loss of knowledge and
capability typically found when we get the next generation of advanced students and the
previous generation graduates. They are committed to taking the theoretical topics we've
been studying and applying them to new domains, providing a basis for interdisciplinary
work at Evergreen that includes Computer Science in a fundamental way, rather than
merely as a tool or an application.

\linebreak 
\linebreak 
\textbf{The ACRL will be used for research and experiments in clustering, operating systems, and languages.  These needs are very extensive and will require that large quantities of information be processed and backed up. 

In order to provide the support needed for the lab, we need a way to deal with servicing the many simultaneous data requests made over the network.  The most effective way to do this is with a single server.  We need hardware that will give us the level of memory, processor power, and necessary diskspace to back up three seperate client architectures in a 20-50 node system.  None of the machines that we currently have are powerful enough to make a single server architecture feasible.}
\linebreak 
\linebreak 
\textbf{
The machine that we are currently interested in purchasing is a PowerEdge R300.  It costs _______ and has ________ specs.   }


PowerEdge R300
2,863.00

1) since the lab is to be used for a)clustering b)os c)language research, each of these needs to be serviced and managed. The most effective way to do this is with a single server. The particular requirements of this server are rather extensive, primarily because a cluster server is servicing a lot of requests simultaneously over the network.
2)None of the hardware or machines that we currently have or anticipate having are powerful enough to make a single server architecture feasible. The benefits of a single server architecture are coherency, simplified redundancy (i.e. only one machine to back up) and generally space and power efficiency. To gain these benefits, we need a machine with enough memory, processor power, and diskspace to sufficiently fulfill the requirements of three separate client architectures in a 20-50 node network.



\section{Expected Project Impact}

\textit{What is the expected impact of the project?  Who
will it affect; how many students and faculty will be involved; what
will students achieve? Specify the exact role that each staff or faculty
member on the grant application will have in the work associated with
the proposal. If the project is successful, what future students and
programs could make use of this innovation?}
\linebreak 
\linebreak 
\textbf{
The ACRL is expected to have an impact on both the computer science and natural science departments.  There is currently a group of 6 students working in conjunction with Sherri Shulman on setting up the lab (ugly sent).  The lab is expected to be in working order by Fall 2010.  Students will interview members of the natural science faculty to determine how a distributed computing lab could be of benefit.  Sherri will continue to oversee the creation and offer her knowledge and support.}  

\linebreak 
\linebreak 
Sherri will be using the lab for her research and courses during the summer as well as the academic schoolyear.  Advanced students enrolled in SOS and Computability would make yearly use of the lab.  There could possibly be places for advanced CSF students who are interested.}

     
\section{Curriculum Enhancement}
\textit{How will this proposal strengthen and enhance the Evergreen curricula?}


\textbf{The ACRL will provide a lab with more computing power than any other place on campus.  Students will have opportunities to explore distributed computing, operating systems, natural languages, and systems administration.  The Computer Center has already allocated funds for a Systems Administration intern for the lab.  In addition to benefiting computer science students, students and faculty from other disciplines would be invited to work with ACRL students to come up with project ideas.  We envision the ACRL creating opportunities for students and faculty members alike.}


\section{Research and Innovation}
\textit{How will this experiment demonstrate educational
innovation?  (How is this proposal bold, innovative, cutting-edge,
and/or too risky to be funded though other channels?)}

\textbf{Although Evergreen currently possesses an Advanced Computing Lab, it is mostly used for instruction and homework completion.  The ACRL would be intended exclusively for advanced students and research.  It is being partially funded through other channels--_________________- and _____________.    
}

\section{Computing Needs}
\textit{Why should this be done with computing equipment?
What alternatives were considered?}

\textbf{Somebody should explain why we need this.}


\linebreak
\linebreak



\section{Plans and Skills}
\textit{How will you carry this project out?  What special
skills will you require to carry out the proposal and how will you
develop these skills?}

\textbf{Students from S.O.S. and CSF have been working on this project since Fall quarter of 2009 and outside parties continue to express interest.  We in the process of setting up the lab and are in a continual process of learning.}


\section{Requirements for Success}
\textit{What is required for this project to be successful?}

\textbf{In order for our project to be successful, we need to ensure that the lab is easy to operate and understand.  Incoming students need to be able to utilize and improve upon the lab without too much excess effort.  We are currently maintaining a wiki and creating a manual explaining how to run and upkeep the lab.}

 
\section{Cost and Needs}
\textit{What is the cost?  Is there any matching funding
excluding program budgets)?  What equipment comparisons have been made?
How is this equipment justified?  Is the software cost a one-time or is
there an annual cost for use?  Are there expectations for continued
maintenance of the hardware and software?}

\textbf{We have a donation of 40 hard drives from St. Martin's and will also be applying for a NSF grant.  The money would be going towards a one-time cost, although future cycles of student groups may write grants for different technology in the future.  Maintenance will be taken care of by the Systems Administration intern position.
}

\section{Work}
\textit{How will you share your work?}

\textbf{We will be sharing our work with the entire scientific community at Evergreen.  Students will be able to get experience desigining scientific software for clients, and computer science faculty will be able to have a place to conduct research.  The ACRL will bring a new type of scientific research to Evergreen and provide a place for students to learn about advanced computing topics.}



\end{document} 
